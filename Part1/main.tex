\documentclass[11pt,a4paper]{article}

% Language setting
\usepackage[english]{babel}
\usepackage{parskip}
\usepackage{float}
\usepackage[utf8]{inputenc}
\usepackage{graphicx}

\usepackage{amsmath}
\usepackage{amssymb}
\usepackage{amsthm}
\usepackage{tikz}

\usepackage{changepage}
\usepackage{xcolor}
\usepackage{soul}
\usepackage{gensymb}
\usepackage{enumitem}
\usepackage{listings}
\lstset{
  basicstyle=\ttfamily\small,
  columns=flexible,
  breaklines=true,
  frame=single,
  xleftmargin=0.04\textwidth,
  captionpos=b
}
\usepackage[letterpaper,top=0.5in,bottom=0.5in,left=0.5in,right=0.5in,marginparwidth=0in]{geometry}

% Useful packages
\usepackage{amsmath}
\makeatletter
\newcommand{\oset}[3][0ex]{%
  \mathrel{\mathop{#3}\limits^{
    \vbox to#1{\kern-1\ex@
    \hbox{$\scriptstyle#2$}\vss}}}}
\makeatother

\usepackage[colorlinks=true, allcolors=blue]{hyperref}
\usepackage{multicol}
\graphicspath{ {Images/} }

% ----------------------------
% Section / Subsection formatting
% ----------------------------
\usepackage{titlesec}
\titleformat{\section}[block]{\centering\Large\bfseries}{\thesection.}{0.5em}{}
\titleformat{\subsection}[block]{\centering\large\bfseries}{\thesubsection}{0.5em}{}

%-------------------------------------------------------------------------
\begin{document}

\break \break \break \break \break 
\begin{center}

{{\Huge Stony Brook University}}                        \break \break
{{\Huge College of Engineering and Applied Sciences}}   \break \break
{\Huge ESE 507} \break \break \break \break \break \break \break \break \break \break \break \break \break 

\end{center}
Hello!

1. Use Synopsys DesignCompiler to synthesize your unpipelined design with INW=16 and
OUTW=64 for a range of different clock frequencies from slow to fast. Adapt the scripts
you used in HW2.

For each frequency you try, record the area, power, the critical path location, and whether
the timing constraint was met or violated. In your report, make a table that shows this data
for each attempted frequency. Make sure you include units on all values you report (here
and everywhere else in the report).

Make graphs that show the relationships you found between clock frequency and both area
and power. Explain the trends that you observed and explain why they occur. (Make two
graphs. On both, show clock frequency on the x-axis; then show area as the y-axis on one
graph and power as the y-axis on the other.) Make sure use graphs that plot both axes
proportionally (like a scatter graph, not a line graph). Only include the design points where
the timing constraint is MET.

For each frequency, give a description in your report of where the critical path is. Don’t
just copy/paste the endpoints from the synthesis report, but explain logically where the
critical path lies in the module. (For example, “the critical path starts at the output of
register [register name], passes through logic that computes [description of the logic], and
ends at the input to register [register name].”)


2. Now, repeat the tasks from question 1 for your pipelined design with the same values of
INW and OUTW. Additionally, answer the following: Did pipelining help make this module
faster? Explain why or why not and show how this is reflected in the synthesis data and
critical path.


3. For the pipelined design with the maximum clock frequency you found, how much energy
would your system consume if it were to process a sequence of 50 sets of input values?

Assume you have to wait until the final output comes out of the system, and don’t forget
that your pipelined design takes more than 50 cycles to compute 50 sets of inputs.

Remember: energy is measured in joules. Power = energy per time. 1 Watt = 1 Joule / 1
second. Use the power obtained from synthesis and your understanding of the time it would
take for your system to fully compute 50 sets of input values.


4. Would the energy you computed in question 3 change if you resynthesized the design
targeting different clock frequencies? Explain and justify your answer. Think carefully
about what changes when you change the target frequency and how those changes affect
the power the system consumes.


5. Make a table that compares the power, area, latency, and throughput of your pipelined and
unpipelined MAC designs (at the maximum clock frequency you previously found) with
INW=16 and OUTW=64. In your report show how you calculated the latency and throughput.
Quantify latency in seconds (or ns), and quantify throughput in terms of MACs per second.
(If needed, review these concepts in the Topic 6 slides.) Based on the trade-offs seen in
your table, explain when it would make sense for a designer to choose the pipelined design
and when it would make sense to use the unpipelined design.


6. Your design is pipelined as much as possible if you assume that you cannot pipeline the
arithmetic units themselves. However, as we discussed in Topic 6, we could also pipeline
the multiplier itself. For example, you can replace the multiplier with one that is pipelined
into more stages. Based on your results to questions 1 and 2, would you expect that deeper
pipelining in the multiplier might help you reach higher clock frequencies? Justify why or
why not.

If you were to pipeline the multiplier in this way, what other changes would you have to
make in your module?

Would pipelining the adder be possible and a good idea? Why or why not?
(Answer this question based on your understanding of the design and your answers to prior
questions. You do not need to modify your design/code to answer this question.)


7. In questions 1 and 2, you always synthesized using the same parameter values of INW=16
and OUTW=64. Here, explore how changing INW and OUTW affects the critical path location
and maximum clock frequency of your pipelined MAC module. Don’t forget: to change
these parameters for synthesis, you should edit the default parameter values in your source
code (mac_pipe.sv).

First, do a set of experiments where you set INW=12 and synthesize three designs with
OUTW=24, 48, and 64.
Next, do a new set of experiments where you set OUTW=48 and synthesize three designs
with INW=8, 16, 24.

Do the clock frequency and critical path location change as OUTW changes? Do they change
when INW changes? Explain what you see and what you learn from it. You should expect
to see differences in the scaling behavior of INW and OUTW. Explain why they behave
differently.

Include the six synthesis reports for this question along with your Final Report.


8. The MAC’s accumulator holds OUTW bits. As you know, if the value stored in the
accumulator grows large enough, it will overflow. That is, OUTW bits may not be enough to
store the resulting number.

Assume INW=5 and OUTW=16. What is the maximum number of MACs your system could
perform while guaranteeing that the accumulator cannot overflow? (Hint: what is the
largest magnitude number you could produce on the multiplier’s output? Then, how many
accumulations would it take for that number to produce an overflow in the accumulator?)

Don’t forget that our values are all signed integers, and don’t forget that the accumulator
can be initialized to a signed INW-bit number. Show your reasoning and justify your
answer.


9. If you worked with a partner, please carefully describe each partner’s contribution to this
part of the project. (If you did not work with a partner, skip this.)



\end{document}